% !TeX spellcheck = en_GB
%
\documentclass[presentation]{beamer}
\mode<presentation>{\usetheme{AMSCesenaPurpleAndGold}}
\setbeamertemplate{bibliography item}{\insertbiblabel}
%%%%%%%%%%%%%%%%%%%%%%%%%%%%%%%%%%%%%%%%%%%%%%%%%%%%%%%%%%%%%%%%%%%%%%%%%%%%%%%%
\usepackage[english]{babel}
\usepackage[utf8]{inputenc}
%
\usepackage{SCO-EXTRAAMAS-2021-talk}

%%%%%%%%%%%%%%%%%%%%%%%%%%%%%%%%%%%%%%%%%%%%%%%%%%%%%%%%%%%%%%%%%%%%%%%%%%%%%%%%
\title[\gridex]{
    % same title of the presented paper
	\gridex: An Algorithm for
	\\
	Knowledge Extraction from Black-Box Regressors
}
%
% \subtitle{Extended Abstract}
%
% same authors order of the presented paper
\author[F. Sabbatini et al.]{
	\emph{Federico Sabbatini} % empth the presenting author
	\and 
	Giovanni Ciatto
	\\
	Andrea Omicini
}
%
\institute[UniBo]{
    Dipartimento di Informatica -- Scienza e Ingegneria (DISI)
    \\
    \textsc{Alma Mater Studiorum} -- Università di Bologna
    \\
    \texttt{
        \{\emph{f.sabbatini}, giovanni.ciatto, andrea.omicini\}@unibo.it % emph the presenting author's email
    }
}
%
\date[EXTRAAMAS, 2021]{
	3\textsuperscript{rd} International Workshop on EXplainable and
	\\
	TRAnsparent AI and Multi-Agent Systems (EXTRAAMAS)
	\\
	May 3, 2021, London, UK (online)
}
%%%%%%%%%%%%%%%%%%%%%%%%%%%%%%%%%%%%%%%%%%%%%%%%%%%%%%%%%%%%%%%%%%%%%%%%%%%%%%%%
\AtBeginSection[]
{
%\\\\\\\\\\\\\\\\\\\\\
\begin{frame}<beamer>[c,noframenumbering]
\frametitle{Next in Line\ldots}
\tableofcontents[sectionstyle=show/shaded,subsectionstyle=hide]
\end{frame}
%\\\\\\\\\\\\\\\\\\\\\
}
\AtBeginSubsection[]
{
%\\\\\\\\\\\\\\\\\\\\\
\begin{frame}<beamer>[shrink,noframenumbering]
    \frametitle{Focus on\ldots}
	\mbox{~}
	\tableofcontents[currentsubsection,sectionstyle=shaded,subsectionstyle=show/shaded]
	\mbox{~}
\end{frame}
%\\\\\\\\\\\\\\\\\\\\\
}
%%%%%%%%%%%%%%%%%%%%%%%%%%%%%%%%%%%%%%%%%%%%%%%%%%%%%%%%%%%%%%%%%%%%%%%%%%%%%%%%
\begin{document}
%%%%%%%%%%%%%%%%%%%%%%%%%%%%%%%%%%%%%%%%%%%%%%%%%%%%%%%%%%%%%%%%%%%%%%%%%%%%%%%%

%\\\\\\\\\\\\\\\\\\\\\
\frame{\titlepage}
%\\\\\\\\\\\\\\\\\\\\\

%===============================================================================
\section{Motivation \& Context} % non il contrario?
%===============================================================================

%\\\\\\\\\\\\\\\\\\\\\
\begin{frame}[c]{Context}
    
    %
    \vfill
    %
    \begin{itemize}
        \item ML-based predictors are more and more used in a wide range of fields to perform different tasks
        %
        \begin{itemize}
            \item[e.g.] pattern detection, image and speech recognition, clustering, classification and regression
        \end{itemize}
        
        \vfill
        
        \item predictors hiding their internal logic are named \emph{black boxes}
        %
        \begin{itemize}
            \item[!] black-box predictors cannot give to the user intelligible outputs or \alert{explanations about their outcomes}
        \end{itemize}
        
        \vfill
        
        \item there are critical applications where the output interpretability is not an option
        %
        \begin{itemize}
        	\item for instance healthcare, finance, etc.
        \end{itemize}
    
    	\vfill
        
        \item[$\Rightarrow$] knowledge extraction mechanisms are required to use black-box models in these critical areas 
        
    \end{itemize}
\end{frame}
%\\\\\\\\\\\\\\\\\\\\\

%\\\\\\\\\\\\\\\\\\\\\
\begin{frame}[c]{Motivation}
	Several methods were developed to extract knowledge from black-box predictors, however:
    %
    \vfill
    %
    \begin{itemize}
        \item most of the algorithms existing in the literature focus exclusively on classification tasks, neglecting regression
        %
        \begin{itemize}
            \item[e.g.] \textsc{Trepan}\ccite{Craven1996}, Rule-extraction-as-learning\ccite{craven1994using}, etc.
        \end{itemize}
        
        \vfill
        
        \item most of the knowledge extraction methods applicable to regression black boxes are strongly constrained
        %
        \begin{itemize}
            \item[e.g.] \textsc{RefAnn}\ccite{setiono2002extraction}, only applicable to ANNs with a single hidden layer
        \end{itemize}
        
        \vfill
        
        \item other procedures may suffer of degrading performance when applied to complex problems
        %
        \begin{itemize}
			\item[e.g.] 
			\iter\ccite{huysmans2006iter} as the number of input features grows
		\end{itemize}
        
    \end{itemize}
\end{frame}
%\\\\\\\\\\\\\\\\\\\\\

%\\\\\\\\\\\\\\\\\\\\\
%\begin{frame}{Some state of the art (optional)}
%
%    Provide relevant information about the state of the art / related works here, possibly with references
%
%\end{frame}
%\\\\\\\\\\\\\\\\\\\\\

%\\\\\\\\\\\\\\\\\\\\\
\begin{frame}{Contribution of the paper}

\begin{block}{The new \gridex algorithm extends \iter}
    \begin{itemize}
    	\item overcoming its major drawbacks
        \item producing shorter rule lists
        \item retaining higher fidelity w.r.t. the underlying black box
    \end{itemize}
\end{block}

\end{frame}
%\\\\\\\\\\\\\\\\\\\\\

%===============================================================================
\section{Theory / modelling / design}
%===============================================================================

%\\\\\\\\\\\\\\\\\\\\\
\begin{frame}%[allowframebreaks]
\frametitle{Theory / modelling / design}

    Provide 2-3 slides discussing the Theory / modelling / design

\end{frame}
%\\\\\\\\\\\\\\\\\\\\\

\section{Case study / Experiments / Results}

%\\\\\\\\\\\\\\\\\\\\\
\begin{frame}%[allowframebreaks]
\frametitle{Case study / Experiments / Results}

    Provide 2-3 slides discussing the Case study / Experiments / Results of the paper

\end{frame}
%\\\\\\\\\\\\\\\\\\\\\

\section{Conclusions \& future works}

%\\\\\\\\\\\\\\\\\\\\\
\begin{frame}%[allowframebreaks]
\frametitle{Conclusions}

\begin{block}{Summing up}
	The new \gridex algorithms is able to:
    %
    \begin{itemize}
        \item overcome the \iter non-exhaustivity
        \item focus only on the interesting regions of the input space
        \item produce shorter rule lists and attain better predictive performances then \iter
    \end{itemize}
\end{block}

\end{frame}

\begin{frame}
	\frametitle{Future works}

\begin{exampleblock}{Future works}
	Our next research efforts will address the remaining limitations of \iter and other algorithms, for instance:
    %
    \begin{itemize}
        \item the inability to handle categorical features
        \item the constant output value of the produced decision rules 
    \end{itemize}
\end{exampleblock}

\end{frame}
%\\\\\\\\\\\\\\\\\\\\\

%===============================================================================
\section*{}
%===============================================================================
\frame{\titlepage}

%===============================================================================
\section*{\bibname}
%===============================================================================

\setbeamertemplate{page number in head/foot}{}
%\\\\\\\\\\\\\\\\\\\\\
\begin{frame}[t,allowframebreaks,noframenumbering]\frametitle{\refname}
% \begin{frame}[c]\frametitle{\refname}
	\footnotesize
%	\scriptsize
    \bibliographystyle{plain}
	\bibliography{SCO-EXTRAAMAS-2021-talk}
\end{frame}
%\\\\\\\\\\\\\\\\\\\\\

%%%%%%%%%%%%%%%%%%%%%%%%%%%%%%%%%%%%%%%%%%%%%%%%%%%%%%%%%%%%%%%%%%%%%%%%%%%%%%%%
\end{document}
%%%%%%%%%%%%%%%%%%%%%%%%%%%%%%%%%%%%%%%%%%%%%%%%%%%%%%%%%%%%%%%%%%%%%%%%%%%%%%%%
